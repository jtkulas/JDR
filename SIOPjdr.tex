% Options for packages loaded elsewhere
\PassOptionsToPackage{unicode}{hyperref}
\PassOptionsToPackage{hyphens}{url}
%
\documentclass[
  english,
  man]{apa6}
\usepackage{lmodern}
\usepackage{amsmath}
\usepackage{ifxetex,ifluatex}
\ifnum 0\ifxetex 1\fi\ifluatex 1\fi=0 % if pdftex
  \usepackage[T1]{fontenc}
  \usepackage[utf8]{inputenc}
  \usepackage{textcomp} % provide euro and other symbols
  \usepackage{amssymb}
\else % if luatex or xetex
  \usepackage{unicode-math}
  \defaultfontfeatures{Scale=MatchLowercase}
  \defaultfontfeatures[\rmfamily]{Ligatures=TeX,Scale=1}
\fi
% Use upquote if available, for straight quotes in verbatim environments
\IfFileExists{upquote.sty}{\usepackage{upquote}}{}
\IfFileExists{microtype.sty}{% use microtype if available
  \usepackage[]{microtype}
  \UseMicrotypeSet[protrusion]{basicmath} % disable protrusion for tt fonts
}{}
\makeatletter
\@ifundefined{KOMAClassName}{% if non-KOMA class
  \IfFileExists{parskip.sty}{%
    \usepackage{parskip}
  }{% else
    \setlength{\parindent}{0pt}
    \setlength{\parskip}{6pt plus 2pt minus 1pt}}
}{% if KOMA class
  \KOMAoptions{parskip=half}}
\makeatother
\usepackage{xcolor}
\IfFileExists{xurl.sty}{\usepackage{xurl}}{} % add URL line breaks if available
\IfFileExists{bookmark.sty}{\usepackage{bookmark}}{\usepackage{hyperref}}
\hypersetup{
  pdftitle={The subjective experience of O*NET work experiences as demands and resources},
  pdfauthor={Alicia Stachowski1, Renata Garcia Prieto Palacios Roji2, \& John Kulas2},
  pdflang={en-EN},
  pdfkeywords={keywords},
  hidelinks,
  pdfcreator={LaTeX via pandoc}}
\urlstyle{same} % disable monospaced font for URLs
\usepackage{graphicx}
\makeatletter
\def\maxwidth{\ifdim\Gin@nat@width>\linewidth\linewidth\else\Gin@nat@width\fi}
\def\maxheight{\ifdim\Gin@nat@height>\textheight\textheight\else\Gin@nat@height\fi}
\makeatother
% Scale images if necessary, so that they will not overflow the page
% margins by default, and it is still possible to overwrite the defaults
% using explicit options in \includegraphics[width, height, ...]{}
\setkeys{Gin}{width=\maxwidth,height=\maxheight,keepaspectratio}
% Set default figure placement to htbp
\makeatletter
\def\fps@figure{htbp}
\makeatother
\setlength{\emergencystretch}{3em} % prevent overfull lines
\providecommand{\tightlist}{%
  \setlength{\itemsep}{0pt}\setlength{\parskip}{0pt}}
\setcounter{secnumdepth}{-\maxdimen} % remove section numbering
% Make \paragraph and \subparagraph free-standing
\ifx\paragraph\undefined\else
  \let\oldparagraph\paragraph
  \renewcommand{\paragraph}[1]{\oldparagraph{#1}\mbox{}}
\fi
\ifx\subparagraph\undefined\else
  \let\oldsubparagraph\subparagraph
  \renewcommand{\subparagraph}[1]{\oldsubparagraph{#1}\mbox{}}
\fi
% Manuscript styling
\usepackage{upgreek}
\captionsetup{font=singlespacing,justification=justified}

% Table formatting
\usepackage{longtable}
\usepackage{lscape}
% \usepackage[counterclockwise]{rotating}   % Landscape page setup for large tables
\usepackage{multirow}		% Table styling
\usepackage{tabularx}		% Control Column width
\usepackage[flushleft]{threeparttable}	% Allows for three part tables with a specified notes section
\usepackage{threeparttablex}            % Lets threeparttable work with longtable

% Create new environments so endfloat can handle them
% \newenvironment{ltable}
%   {\begin{landscape}\centering\begin{threeparttable}}
%   {\end{threeparttable}\end{landscape}}
\newenvironment{lltable}{\begin{landscape}\centering\begin{ThreePartTable}}{\end{ThreePartTable}\end{landscape}}

% Enables adjusting longtable caption width to table width
% Solution found at http://golatex.de/longtable-mit-caption-so-breit-wie-die-tabelle-t15767.html
\makeatletter
\newcommand\LastLTentrywidth{1em}
\newlength\longtablewidth
\setlength{\longtablewidth}{1in}
\newcommand{\getlongtablewidth}{\begingroup \ifcsname LT@\roman{LT@tables}\endcsname \global\longtablewidth=0pt \renewcommand{\LT@entry}[2]{\global\advance\longtablewidth by ##2\relax\gdef\LastLTentrywidth{##2}}\@nameuse{LT@\roman{LT@tables}} \fi \endgroup}

% \setlength{\parindent}{0.5in}
% \setlength{\parskip}{0pt plus 0pt minus 0pt}

% \usepackage{etoolbox}
\makeatletter
\patchcmd{\HyOrg@maketitle}
  {\section{\normalfont\normalsize\abstractname}}
  {\section*{\normalfont\normalsize\abstractname}}
  {}{\typeout{Failed to patch abstract.}}
\patchcmd{\HyOrg@maketitle}
  {\section{\protect\normalfont{\@title}}}
  {\section*{\protect\normalfont{\@title}}}
  {}{\typeout{Failed to patch title.}}
\makeatother
\shorttitle{Title}
\keywords{keywords\newline\indent Word count: X}
\DeclareDelayedFloatFlavor{ThreePartTable}{table}
\DeclareDelayedFloatFlavor{lltable}{table}
\DeclareDelayedFloatFlavor*{longtable}{table}
\makeatletter
\renewcommand{\efloat@iwrite}[1]{\immediate\expandafter\protected@write\csname efloat@post#1\endcsname{}}
\makeatother
\usepackage{lineno}

\linenumbers
\usepackage{csquotes}
\ifxetex
  % Load polyglossia as late as possible: uses bidi with RTL langages (e.g. Hebrew, Arabic)
  \usepackage{polyglossia}
  \setmainlanguage[]{english}
\else
  \usepackage[shorthands=off,main=english]{babel}
\fi
\ifluatex
  \usepackage{selnolig}  % disable illegal ligatures
\fi
\newlength{\cslhangindent}
\setlength{\cslhangindent}{1.5em}
\newlength{\csllabelwidth}
\setlength{\csllabelwidth}{3em}
\newenvironment{CSLReferences}[2] % #1 hanging-ident, #2 entry spacing
 {% don't indent paragraphs
  \setlength{\parindent}{0pt}
  % turn on hanging indent if param 1 is 1
  \ifodd #1 \everypar{\setlength{\hangindent}{\cslhangindent}}\ignorespaces\fi
  % set entry spacing
  \ifnum #2 > 0
  \setlength{\parskip}{#2\baselineskip}
  \fi
 }%
 {}
\usepackage{calc}
\newcommand{\CSLBlock}[1]{#1\hfill\break}
\newcommand{\CSLLeftMargin}[1]{\parbox[t]{\csllabelwidth}{#1}}
\newcommand{\CSLRightInline}[1]{\parbox[t]{\linewidth - \csllabelwidth}{#1}\break}
\newcommand{\CSLIndent}[1]{\hspace{\cslhangindent}#1}

\title{The subjective experience of O*NET work experiences as demands and resources}
\author{Alicia Stachowski\textsuperscript{1}, Renata Garcia Prieto Palacios Roji\textsuperscript{2}, \& John Kulas\textsuperscript{2}}
\date{}


\affiliation{\vspace{0.5cm}\textsuperscript{1} University of Wisconsin - Stout\\\textsuperscript{2} Montclair State University}

\abstract{
O*NET work characteristics were rated in terms of relevance, perception of demand, and perception as resource.
}



\begin{document}
\maketitle

The job demands-resources model (Demerouti, Bakker, Nachreiner, \& Schaufeli, 2001) and later job demands-resources theory (A. B. Bakker \& Demerouti, 2017) have inspired a plethora a study on the process and experience of job stress and employee motivation in recent decades. In the current project, we draw attention to a basic question regarding a key assumption we make regarding this process - that of the objective nature of job characteristics as either demands or resources. The major contribution of this project is to document whether job context and characteristics (pulled from O*Net) can simultaneously be classified as resources and as demands. We further present descriptive information regarding which job context and characteristics are rated the highest across jobs.

\hypertarget{the-job-demands-resources-theory}{%
\subsection{The Job demands-Resources Theory}\label{the-job-demands-resources-theory}}

The job demands-resources theory is an extension of the well-known job demands-resources model put forth by Demerouti and colleagues in 2001 (Demerouti, Bakker, Nachreiner, \& Schaufeli, 2001). The job demands-resources model had been so heavily studied that a number of meta-analyses have been possible (e.g., (Crawford, LePine, \& Rich, 2010);
(Halbesleben, 2010); (Nahrgang, Morgeson, \& Hofmann, 2011)). The theory generated by the model integrates both the job design and job stress literatures to help explain the conditions under which a job would result in employee stress vs.~motivation (A. B. Bakker \& Demerouti, 2014). Per the job demands-resources theory, both work environment and job characteristics can be modeled via job demands and resources. Demerouti, Bakker, Nachreiner, and Schaufeli (2001) define job demands broadly as components of a job that require sustained effort, and as such, produce psychological or physiological strain (e.g., high work pressure is frequently cited as a common demand). Resources, on the other hand, are physical, psychological, social, or organizational aspects of the job that may help an employee achieve work goals, reduce job demands, or promote personal growth and development (Demerouti, Bakker, Nachreiner, \& Schaufeli, 2001).
Experiencing an element of one's job as a resource or demand activates one of two distinct processes: either health impairment (demands) or motivation (resources; (A. B. Bakker \& Demerouti, 2014). Job characteristics perceived to be demanding are effortful are frequently associated with negative outcomes such as exhaustion (e.g., A. Bakker, Demerouti, \& Schaufeli, 2003). On the other hand, job characteristics perceived as resources (fulfil psychological needs) are associated with positive organizational outcomes like engagement and motivation (A. B. Bakker, Hakanen, Demerouti, \& Xanthopoulou, 2007).

\hypertarget{objective-vs.-subjective-nature-of-demands-and-resources-the-role-of-appraisal}{%
\subsection{Objective vs.~Subjective Nature of Demands and Resources: The Role of Appraisal}\label{objective-vs.-subjective-nature-of-demands-and-resources-the-role-of-appraisal}}

Searle and Auton (2015) note that the majority of the research on workplace demands is based on apriori classifications of demands. However, the stress experience, or process, described early on by Lazarus and Folkman (1984) is grounded in the assumption that individual appraisals of stressors/demands vary. Their transactional theory or stress and coping states that people continuously appraise stimuli in their environments. An appraisal is the cognitive process whereby meaning is assigned to a stimulus. If a stimulus is appraised as a stressor (threat, challenge, potentially harmful), emotional distress leads to coping of some kind. This action to cope is also associated with another appraisal about the outcome itself and the process continues if the outcomes is not appraised as favorable (Lazarus \& Folkman, 1984). The stress appraisal process suggests that classifying a job characteristic or environmental condition as an objective demand or resource might be in error.
We next consider the (limited) empirical evidence on this topic. First, some relatively recent research suggests that job demands and resources may not be universally appraised or assigned as such. Starting with job demands, Webster, Beehr, and Love (2011), for example, studied workload, role ambiguity, and role conflict demands, and found while that each could be appraised primarily as challenges or hindrances demands, they could also simultaneously be perceived as being both a challenge and hinderance to different degrees. While their study did include resources, it nonetheless points to individual difference on how people perceive stressors at work. Although part of a much larger study on retirement, Sonnega, Helppie-McFall, Hudomiet, Willis, and Fisher (2018) compared self-reported (subjective) ratings of degree of physical demand, stress, and need for intense concentration from the Health and Retirement Study with objective ratings from O*Net. Correlations physical demand (r = .52), stress (r = .10), and need for intense concentration (r = .14), again suggesting perhaps that our objective ratings of job demands (and resources) may be subject to a greater level of individual difference than assumed. Next considering resources, Schmitz, McCluney, Sonnega, and Hicken (2019) captured subjective and objective resources in their study of retirement also. Correlations of composite variables for the resources of autonomy (r = .12), recognition of work (r = .07), decision freedom (r = .08), and advancement (r = -.01), while significant, certainly do not reflect high levels of overlap.
We do acknowledge as well, that demands and resources are not necessarily consistent across days, or seasons, for many employees. Downes, Reeves, McCormick, Boswell, and Butts (2021) meta-analysis addresses this reality in depth, although it is beyond the scope of this project.

\hypertarget{current-study-and-hypotheses}{%
\subsection{Current Study and Hypotheses}\label{current-study-and-hypotheses}}

The current study aims to explore the degree to which job context and job characteristic items from O*Net are considered demands and resources. Given theoretical and empirical findings, it seems quite plausible that our apriori assignment of job elements to a ``demand'' or ``resource'' category may be too simplistic. We aim to document a list of the highest rated demands and resources, as well as information on overlap of job characteristics as demands and resources, in addition to addressing the following predictions.

\begin{quote}
\emph{Hypothesis 1}: There is
\end{quote}

\hypertarget{current-study-and-research-questions-for-other-studies-notes}{%
\subsection{Current Study and Research Questions for other studies + notes}\label{current-study-and-research-questions-for-other-studies-notes}}

A. B. Bakker and Demerouti (2017) state that, ``\ldots research has shown that challenge demands may be experienced as hindrance demands (and vice versa) depending on the context'' (p.~278). We extend this acknowlegement by investigating whether some characteristics of work may also vacillate between demand and \emph{resource}.

\hypertarget{study-2-introduction-correlates-with-engagement-and-stress}{%
\section{Study 2 Introduction: Correlates with Engagement and Stress}\label{study-2-introduction-correlates-with-engagement-and-stress}}

Research on the job demands-resources model (Demerouti, Bakker, Nachreiner, \& Schaufeli, 2001) and later job demands-resources theory (A. B. Bakker \& Demerouti, 2017) highlight the importance of work characteristics on the experience of motivation and strain, which clearly have an impact on job performance. In this paper, we extend this critical research to that of the distinction between challenge and hinderance demands (and resource) in the workplace, and how they relate to two important organizational outcomes: engagement and stress. Prior to presenting the current study in detail, we provide a brief overview of the relevant theories and relevant empirical work on this topic.

\#\#The Job demands-Resources Theory

The overarching context for this study is that of the job demands-resources theory, which is an expansion of the well-studied job demands-resources model (Demerouti, Bakker, Nachreiner, \& Schaufeli, 2001). One of the major advantages of the job demands-resources theory is that it allows us to model both work environment and job characteristics via job resources and demands. \emph{Resources} include physical, psychological, social, or organizational aspects of the job that may help an employee achieve work goals, reduce job demands, or promote personal growth and development (Demerouti, Bakker, Nachreiner, \& Schaufeli, 2001). In contrast, demands include components of a job that require sustained effort, and as such, produce psychological or physiological strain (e.g., high work pressure is frequently cited as a common demand; Demerouti, Bakker, Nachreiner, and Schaufeli (2001)).

Cognitively, the perception of an element of one's job as a resource or demand activates one of two distinct processes: either health impairment (resulting from demands) or motivation (resulting from resources) (A. B. Bakker \& Demerouti, 2014). Pertinent to the current study, demanding job characteristics are frequently often associated with negative outcomes (e.g., A. Bakker, Demerouti, \& Schaufeli, 2003), whereas job characteristics deemed resources have been associated with positive organizational outcomes like engagement and motivation (A. B. Bakker, Hakanen, Demerouti, \& Xanthopoulou, 2007).

\hypertarget{the-essential-role-of-appraisal}{%
\subsection{The Essential Role of Appraisal}\label{the-essential-role-of-appraisal}}

As implied in the last paragraph, job context and characteristics are ``assigned'' or appraised as demands or resources. Although some research on job demands in particular is based on apriori classifications of demands (Searle \& Auton, 2015), the classification of a work characteristic as a demand or resource is largely subjective by nature (e.g., an employee could most certainly perceive being a public figure as a resource or as a demand. The stress process speaks to how such individual difference in appraisal is possible. Lazarus and Folkman (1984) presented the transactional theory of stress and coping, which states that people cognitively appraise stimuli in their environments on a continuous basis. Via this process, meaning is assigned to stimuli -- if appraised as threatening, challenging, or possibly harmful, the resulting emotional distress initiates coping. The cycle of appraisal then continues based on the action to cope with the stressor (Lazarus \& Folkman, 1984).

\hypertarget{the-challenge-hinderance-framework}{%
\subsection{The Challenge-Hinderance Framework}\label{the-challenge-hinderance-framework}}

Although there is a tendency to attach a negative connotation to the word ``stress,'' Selye (1936) defined stress as a response to change, which is quite non-specific. We return to the employed public figure for this next section. It is quite probable that two employees would be called upon to serve as a spokesperson for their organization in a time of need. One may appraise the circumstance as an opportunity to positively influence others, while the other may plausibly feel paralyzed by the task. Cavanaugh, Boswell, Roehling, and Boudreau (2000) delineated between two forms of demands -- that of \emph{challenge} and \emph{hinderance} demands. Challenge demands promote mastery, personal growth, and future gains. Hinderance demands, in contrast, inhibit growth, learning and goal achievement. This particular distinction has been of value in determining what demands are related to various outcomes, whereby challenge stressors are typically associated with positive outcomes, and hinderance stressors, negative outcomes (e.g., Cavanaugh, Boswell, Roehling, and Boudreau (2000)). However, one of the key questions we need to ask as researchers pertains to the very basic consideration of appraisals.

We next consider the empirical evidence on this topic. The first obvious question is whether people perceive demands and challenges vs.~hinderances, or whether all demands are under a larger ``demands'' category. Webster, Beehr, and Love (2011) approached this question with three common workplace demands: workload, role ambiguity, and role conflict. They found while that each could be appraised primarily as challenges or hindrances demands, they could also simultaneously be perceived as being both a challenge and hinderance to different degrees. While their study did include resources, it nonetheless points to the possibility that demands might be differentially appraised. Cavanaugh, Boswell, Roehling, and Boudreau (2000), in a study of managers, found that challenge demands were positively related to job satisfaction and negatively related to job search behaviors, while hinderance demands demonstrated the opposite pattern.

\hypertarget{notes-on-which-other-studies-to-read-and-add-next}{%
\subsubsection{Notes on which other studies to read and add next}\label{notes-on-which-other-studies-to-read-and-add-next}}

A. B. Bakker and Sanz-Vergel (2013) Weekly work engagement and flourishing: The role of
hindrance and challenge job demands
@ crawford2010linking Crawford, E. R., Lepine, J. A., \& Rich, B. L. (2010). Linking job demands and resources to employee engagement and burnout: A theoretical extension and meta-analytic test. The Journal of Applied Psychology, 95(5), 834--48. \url{doi:10.1037/a0019364}
@ lepine2004challenge LePine, J. A., LePine, M. A., \& Jackson, C. L. (2004). Challenge and hindrance stress: relationships with exhaustion, motivation to learn, and learning performance. The Journal of Applied Psychology, 89(5), 883--91. \url{doi:10.1037/0021-9010.89.5.883}
Podsakoff, LePine, and LePine (2007) Podsakoff, N. P., LePine, J. A., \& LePine, M. A. (2007). Differential challenge stressorhindrance stressor relationships with job attitudes, turnover intentions, turnover, and 43 withdrawal behavior: a meta-analysis. The Journal of Applied Psychology, 92(2), 438--54. \url{doi:10.1037/0021-9010.92.2.438}
Look at the resources in the following paper as well:
O'Brien and Beehr (2019) O'Brien, K. E., \& Beehr, T. A. (2019). So far, so good: Up to now, the challenge--hindrance framework describes a practical and accurate distinction. Journal of Organizational Behavior, 40(8), 962-972.

\hypertarget{current-study-and-hypotheses-1}{%
\subsection{Current Study and Hypotheses}\label{current-study-and-hypotheses-1}}

Given the abundance of theoretical and empirical support for the connection between resources and positive organizational outcomes (cites), and between demands and negative resources, we sought to explore whether or not the appraisal of a demand as a challenge or hinderance would be related \emph{differently} to two organizational outcomes: engagement DEFINE THESE (a positive affective experience) and workplace stress (a negative affective experience). Drawing on the job demands-resources theory we propose that job elements appraised as ``challenge demands'' (i.e., promote mastery, personal growth, and future gains) would activate (be related to) a positive state -- that of engagement. In contrast, elements of one's job appraised as a hinderance demand (i.e., inhibit growth, learning and goal achievement) would activate a negative state -- here, stress.\\
Hypothesis 1a: Job characteristics appraised as resources will be positively associated with engagement.

\begin{quote}
\emph{Hypothesis 1b}: Job characteristics appraised as resources will be negatively associated with stress.
\end{quote}

\begin{quote}
\emph{Hypothesis 2a}: Job characteristics appraised as challenge demands will be positively associated with engagement.
\end{quote}

\begin{quote}
\emph{Hypothesis 2b}: Job characteristics appraised as challenge demands will be negatively associated with stress.
\end{quote}

\begin{quote}
\emph{Hypothesis 3a}: Job characteristics appraised as hinderance demands will be negatively associated with engagement.
\end{quote}

\begin{quote}
\emph{Hypothesis 3b}: Job characteristics appraised as hinderance demands will be positively associated with stress.
\end{quote}

\hypertarget{methods}{%
\section{Methods}\label{methods}}

\hypertarget{study-1}{%
\subsection{Study 1}\label{study-1}}

top 15 demands and resources, divided by skilled versus knowledge workers,

\hypertarget{study-2}{%
\subsection{Study 2}\label{study-2}}

burnout and stress components (correlations),

\hypertarget{study-3}{%
\subsection{Study 3}\label{study-3}}

integration of JDR with O*Net categories (morphs into descriptives).

We report how we determined our sample size, all data exclusions (if any), all manipulations, and all measures in the study.

\hypertarget{participants}{%
\subsection{Participants}\label{participants}}

\hypertarget{material}{%
\subsection{Material}\label{material}}

Job Characteristics (O*Net)
Job Resources
Job Demands (Hinderance and Challenge)
Stress: Copenhagen Psychosocial Questionnaire
Engagement
Demographics
\#\# Procedure

Qualtrics panel

\hypertarget{data-analysis}{%
\subsection{Data analysis}\label{data-analysis}}

We used R {[}Version 4.0.3; R Core Team (2020){]} and the R-package \emph{papaja} {[}Version 0.1.0.9997; Aust and Barth (2020){]} for all our analyses.

\hypertarget{results}{%
\section{Results}\label{results}}

\hypertarget{discussion}{%
\section{Discussion}\label{discussion}}

\newpage

\hypertarget{references}{%
\section{References}\label{references}}

\begingroup
\setlength{\parindent}{-0.5in}
\setlength{\leftskip}{0.5in}

\hypertarget{refs}{}
\begin{CSLReferences}{1}{0}
\leavevmode\hypertarget{ref-R-papaja}{}%
Aust, F., \& Barth, M. (2020). \emph{{papaja}: {Create} {APA} manuscripts with {R Markdown}}. Retrieved from \url{https://github.com/crsh/papaja}

\leavevmode\hypertarget{ref-bakker2014job}{}%
Bakker, A. B., \& Demerouti, E. (2014). Job demands--resources theory. \emph{Wellbeing: A Complete Reference Guide}, 1--28.

\leavevmode\hypertarget{ref-bakker2017job}{}%
Bakker, A. B., \& Demerouti, E. (2017). Job demands--resources theory: Taking stock and looking forward. \emph{Journal of Occupational Health Psychology}, \emph{22}(3), 273.

\leavevmode\hypertarget{ref-bakker2007job}{}%
Bakker, A. B., Hakanen, J. J., Demerouti, E., \& Xanthopoulou, D. (2007). Job resources boost work engagement, particularly when job demands are high. \emph{Journal of Educational Psychology}, \emph{99}(2), 274.

\leavevmode\hypertarget{ref-bakker2007job}{}%
Bakker, A. B., Hakanen, J. J., Demerouti, E., \& Xanthopoulou, D. (2007). Job resources boost work engagement, particularly when job demands are high. \emph{Journal of Educational Psychology}, \emph{99}(2), 274.

\leavevmode\hypertarget{ref-bakker2013weekly}{}%
Bakker, A. B., \& Sanz-Vergel, A. I. (2013). Weekly work engagement and flourishing: The role of hindrance and challenge job demands. \emph{Journal of Vocational Behavior}, \emph{83}(3), 397--409.

\leavevmode\hypertarget{ref-bakker2003dual}{}%
Bakker, A., Demerouti, E., \& Schaufeli, W. (2003). Dual processes at work in a call centre: An application of the job demands--resources model. \emph{European Journal of Work and Organizational Psychology}, \emph{12}(4), 393--417.

\leavevmode\hypertarget{ref-bakker2003dual}{}%
Bakker, A., Demerouti, E., \& Schaufeli, W. (2003). Dual processes at work in a call centre: An application of the job demands--resources model. \emph{European Journal of Work and Organizational Psychology}, \emph{12}(4), 393--417.

\leavevmode\hypertarget{ref-cavanaugh2000empirical}{}%
Cavanaugh, M. A., Boswell, W. R., Roehling, M. V., \& Boudreau, J. W. (2000). An empirical examination of self-reported work stress among US managers. \emph{Journal of Applied Psychology}, \emph{85}(1), 65.

\leavevmode\hypertarget{ref-crawford2010linking}{}%
Crawford, E. R., LePine, J. A., \& Rich, B. L. (2010). Linking job demands and resources to employee engagement and burnout: A theoretical extension and meta-analytic test. \emph{Journal of Applied Psychology}, \emph{95}(5), 834.

\leavevmode\hypertarget{ref-crawford2010linking}{}%
Crawford, E. R., LePine, J. A., \& Rich, B. L. (2010). Linking job demands and resources to employee engagement and burnout: A theoretical extension and meta-analytic test. \emph{Journal of Applied Psychology}, \emph{95}(5), 834.

\leavevmode\hypertarget{ref-crawford2010linking}{}%
Crawford, E. R., LePine, J. A., \& Rich, B. L. (2010). Linking job demands and resources to employee engagement and burnout: A theoretical extension and meta-analytic test. \emph{Journal of Applied Psychology}, \emph{95}(5), 834.

\leavevmode\hypertarget{ref-demerouti2001job}{}%
Demerouti, E., Bakker, A. B., Nachreiner, F., \& Schaufeli, W. B. (2001). The job demands-resources model of burnout. \emph{Journal of Applied Psychology}, \emph{86}(3), 499.

\leavevmode\hypertarget{ref-demerouti2001job}{}%
Demerouti, E., Bakker, A. B., Nachreiner, F., \& Schaufeli, W. B. (2001). The job demands-resources model of burnout. \emph{Journal of Applied Psychology}, \emph{86}(3), 499.

\leavevmode\hypertarget{ref-downes2021incorporating}{}%
Downes, P. E., Reeves, C. J., McCormick, B. W., Boswell, W. R., \& Butts, M. M. (2021). Incorporating job demand variability into job demands theory: A meta-analysis. \emph{Journal of Management}, \emph{47}(6), 1630--1656.

\leavevmode\hypertarget{ref-halbesleben2010meta}{}%
Halbesleben, J. R. (2010). A meta-analysis of work engagement: Relationships with burnout, demands, resources, and consequences. \emph{Work Engagement: A Handbook of Essential Theory and Research}, \emph{8}(1), 102--117.

\leavevmode\hypertarget{ref-halbesleben2010meta}{}%
Halbesleben, J. R. (2010). A meta-analysis of work engagement: Relationships with burnout, demands, resources, and consequences. \emph{Work Engagement: A Handbook of Essential Theory and Research}, \emph{8}(1), 102--117.

\leavevmode\hypertarget{ref-lazarus1984stress}{}%
Lazarus, R. S., \& Folkman, S. (1984). \emph{Stress, appraisal, and coping}. Springer publishing company.

\leavevmode\hypertarget{ref-nahrgang2011safety}{}%
Nahrgang, J. D., Morgeson, F. P., \& Hofmann, D. A. (2011). Safety at work: A meta-analytic investigation of the link between job demands, job resources, burnout, engagement, and safety outcomes. \emph{Journal of Applied Psychology}, \emph{96}(1), 71.

\leavevmode\hypertarget{ref-nahrgang2011safety}{}%
Nahrgang, J. D., Morgeson, F. P., \& Hofmann, D. A. (2011). Safety at work: A meta-analytic investigation of the link between job demands, job resources, burnout, engagement, and safety outcomes. \emph{Journal of Applied Psychology}, \emph{96}(1), 71.

\leavevmode\hypertarget{ref-o2019so}{}%
O'Brien, K. E., \& Beehr, T. A. (2019). So far, so good: Up to now, the challenge--hindrance framework describes a practical and accurate distinction. \emph{Journal of Organizational Behavior}, \emph{40}(8), 962--972.

\leavevmode\hypertarget{ref-podsakoff2007differential}{}%
Podsakoff, N. P., LePine, J. A., \& LePine, M. A. (2007). Differential challenge stressor-hindrance stressor relationships with job attitudes, turnover intentions, turnover, and withdrawal behavior: A meta-analysis. \emph{Journal of Applied Psychology}, \emph{92}(2), 438.

\leavevmode\hypertarget{ref-R-base}{}%
R Core Team. (2020). \emph{R: A language and environment for statistical computing}. Vienna, Austria: R Foundation for Statistical Computing. Retrieved from \url{https://www.R-project.org/}

\leavevmode\hypertarget{ref-schmitz_interpreting_2019}{}%
Schmitz, L. L., McCluney, C. L., Sonnega, A., \& Hicken, M. T. (2019). Interpreting {Subjective} and {Objective} {Measures} of {Job} {Resources}: {The} {Importance} of {Sociodemographic} {Context}. \emph{International Journal of Environmental Research and Public Health}, \emph{16}(17), 3058. \url{https://doi.org/10.3390/ijerph16173058}

\leavevmode\hypertarget{ref-searle2015merits}{}%
Searle, B. J., \& Auton, J. C. (2015). The merits of measuring challenge and hindrance appraisals. \emph{Anxiety, Stress, \& Coping}, \emph{28}(2), 121--143.

\leavevmode\hypertarget{ref-selye1936syndrome}{}%
Selye, H. (1936). A syndrome produced by diverse nocuous agents. \emph{Nature}, \emph{138}(3479), 32--32.

\leavevmode\hypertarget{ref-sonnega_comparison_2018}{}%
Sonnega, A., Helppie-McFall, B., Hudomiet, P., Willis, R. J., \& Fisher, G. G. (2018). A {Comparison} of {Subjective} and {Objective} {Job} {Demands} and {Fit} {With} {Personal} {Resources} as {Predictors} of {Retirement} {Timing} in a {National} {U}.{S}. {Sample}. \emph{Work, Aging and Retirement}, \emph{4}(1), 37--51. \url{https://doi.org/10.1093/workar/wax016}

\leavevmode\hypertarget{ref-webster2011extending}{}%
Webster, J. R., Beehr, T. A., \& Love, K. (2011). Extending the challenge-hindrance model of occupational stress: The role of appraisal. \emph{Journal of Vocational Behavior}, \emph{79}(2), 505--516.

\end{CSLReferences}

\endgroup


\end{document}
